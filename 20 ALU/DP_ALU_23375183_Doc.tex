\documentclass{article}
\usepackage{graphicx} % Required for inserting images
\usepackage[T2A]{fontenc}  % For Cyrillic font encoding
\usepackage[utf8]{inputenc}  % For UTF-8 input encoding

\title{Arithmetic Logic Unit (ALU)}
\author{Kyara McWilliam 23375183}
\date{}

\begin{document}

\maketitle

Test Cases
\hfill \break

\begin{tabular}{ |c|c|c|c|c|c|c|c| }
\hline
Case & C & S0 & S1 & S2 & Result & CV & Meaning \\
\hline
A & 0 & 0 & 0 & 0 & 23375183 & 00 & A \\
B & 1 & 0 & 0 & 0 & 23375184 & 00 & A + 1 \\
C & 0 & 1 & 0 & 0 & 23440718 & 00 & A + B \\
D & 1 & 1 & 0 & 0 & 23440719 & 00 & A + B + 1 \\
E & 0 & 0 & 1 & 0 & 23309647 & 10 & A + 1sC B \\
F & 1 & 0 & 1 & 0 & 23309648 & 10 & A + 1sC B + 1 \\
G & 0 & 1 & 1 & 0 & 23375182 & 10 & A - 1 \\
H & 1 & 1 & 1 & 0 & 23375183 & 10 & A - 1 + 1 \\
I & 0 & 0 & 0 & 1 & 44367 & 00 & A AND B \\
J & 0 & 1 & 0 & 1 & 23396351 & 00 & A OR B \\
K & 0 & 0 & 1 & 1 & 23351984 & 10 & A XOR B \\
L & 0 & 1 & 1 & 1 & 4271592112 & 10 & NOT A \\
\hline
\end{tabular}
\hfill \break \break
The testbench implements every ALU operation, using the value 23375183 in A and the value 65535 in B. 
\hfill \break \break
S2 high signals a logical operation (OR, XOR, AND, NOT / 1sC),  while S2 low signals an arithmetic operation. The precise operation being conducted within the modes is in turn defined by S1 and S0, with C being used in arithmetic mode to add 1.
\hfill \break \break
This results in a total potential number of operations of 12, with 8 from the permutations of S2 low, S1, S0 and C and 4 from the permutations of S2 high, S1 and S0. Any use of C in logical operations has no defined effect on the result.
\end{document}
