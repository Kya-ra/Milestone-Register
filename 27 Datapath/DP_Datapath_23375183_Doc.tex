\documentclass{article}
\usepackage{graphicx} % Required for inserting images
\usepackage[T2A]{fontenc}  % For Cyrillic font encoding
\usepackage[utf8]{inputenc}  % For UTF-8 input encoding

\title{Datapath}
\author{Kyara McWilliam 23375183}
\date{}

\begin{document}

\begin{tabular}{ |c|c|c| }
\hline
Case & Purpose & Expected Value \\
\hline
A & Load Q0 & 23375183 in Q0 \\
B & Load Q1 & 23375182 in Q1\\
C & Load Q2 & 23375181 in Q2\\
D & Load Q3 & 23375180 in Q3\\
E & Load Q4 & 23375179 in Q4\\
F & Load Q5 & 23375178 in Q5\\
G & Load Q6 & 23375177 in Q6\\
H & Load Q7 & 23375176 in Q7\\
I & Load Q8 & 23375175 in Q8\\
J & Load Q9 & 23375174 in Q9\\
K & Load Q10 & 23375173 in Q10\\
L & Load Q11 & 23375172 in Q11\\
M & Load Q12 & 23375171 in Q12\\
N & Load Q13 & 23375170 in Q13\\
O & Load Q14 & 23375169 in Q14\\
P & Load Q15 & 23375168 in Q15\\
Q & Load Q16 & 23375167 in Q16\\
R & Load Q17 & 23375166 in Q17\\
S & Load Q18 & 23375165 in Q18\\
T & Load Q19 & 23375164 in Q19\\
U & Load Q20 & 23375163 in Q20\\
V & Load Q21 & 23375162 in Q21\\
W & Load Q22 & 23375161 in Q22\\
X & Load Q23 & 23375160 in Q23\\
Y & Load Q24 & 23375159 in Q24\\
Z & Load Q25 & 23375158 in Q25\\
$\alpha$ & Load Q26 & 23375157 in Q26\\
$\beta$ & Load Q27 & 23375156 in Q27\\
$\gamma$ & Load Q28 & 23375155 in Q28\\
$\delta$ & Load Q29 & 23375154 in Q29\\
$\epsilon$ & Load Q30 & 23375153 in Q30\\
$\zeta$ & Load Q31 & 23375152 in Q31\\
$\eta$ & Load QT1 & 23375151 in QT1\\
$\theta$ & Load QT2 & 23375150 in QT2\\
$\iota$ & Load QT3 & 23375149 in QT3\\
$\kappa$ & Load QT4 & 23375148 in QT4\\
$\lambda$ & Load QT5 & 23375147 in QT5\\
\hline
\end{tabular}
\begin{tabular}{ |c|c|c| }
\hline
Case & Purpose & Expected Value \\
\hline
$\mu$ & Load QT6 & 23375146 in QT6\\
$\nu$ & Load QT7 & 23375145 in QT7\\
$\xi$ & Load QT8 & 23375144 in QT8\\
o & Load QT9 & 23375143 in QT9\\
$\pi$ & Load QT10 & 23375142 in QT10\\
$\rho$ & Load QT11 & 23375141 in QT11\\
$\sigma$ & Load QT12 & 23375140 in QT12\\
$\varsigma$ & Load QT13 & 23375139 in QT13\\
$\tau$ & Load QT14 & 23375138 in QT14\\
$\upsilon$ & Load QT15 & 23375137 in QT15\\
$\phi$ & Check RW & No Change \\
$\chi$ & A + 1sC B & 4026531849 in Q3 \\
$\psi$ & A OR B & 23375231 in Q3 \\
$\omega$ & A + B + 1 & 46750341 in Q3 \\
© & A XOR B & 122 in Q3 \\
ß & A + B & 46750340 in Q3 \\
æ & slB & 46750330 in Q3 \\
! & A (FS 00000) & 23375175 in Q3 \\
@ & B & 23375165 in Q3 \\
£ & A + 1 & 23375176 in Q3 \\
€ & A + 1sC B + 1 & 4026531850 in Q3 \\
Б & A - 1 & 23375174 in Q3 \\
Г & 1sC A & 4271592120 in Q3 \\
Д & A AND B & 23375109 in Q3 \\
Є & srB & 11687582 in Q3 \\
Ж & A (FS 00111) & 23375175 in Q3 \\
З & A + 1sc B & 4294967287 in Q3 \\
И & A OR B & 23375183 in Q3 \\
Љ & A + B + 1 & 46750359 in Q3 \\
Њ & A XOR B & 8 in Q3 \\
Ћ & A + B & 46750358 in Q3 \\
Ч & slB & 46750366 in Q3 \\
Ш & B & 23375183 in Q3 \\
Ъ & A + 1sc B + 1 & 4294967288 in Q3 \\
Ы & A AND B & 23375175 in Q3 \\
Ю & SrB & 11687591 in Q3 \\
Я & Reset & 0 in Q0 - QT15 \\
\hline
\end{tabular}
\\

\newpage
\maketitle
TD01a covers register loading operations, cases A through $\phi$ \\
TD01b covers functional unit operations with source register output B, cases $\chi$ through Ж \\
TD01c covers functional unit operations with IR\_IN and the reset, cases З through Я \\
\\
The testbench implements the order of test cases for a student number ending in a 3. First, using Q8 + Q18 as sources, then using Q8 + IR\_IN. As the outputs are specified, output of the circuit is shown on Q3.\\
This means operations are performed on 23375175 (SID - 8) and 23375183 [B]. \\
For IR\_IN, only operations involving the value B are used\\ 
The worst case propagation delay is 202ns, so the period used is 250ns (equivalent to a 4MHz processor)\\ \\
Cases A through $\upsilon$ handle loading the registers, then $\phi$ tests the RW function. $\chi$ through Ю test the functional unit, then Я tests the reset function.\\ \\
FS is composed of 5 signals; S3, S2, S1, S0 and C, from most to least significant respectively. S3 high signals a shift, S2 high signals a logical operation (OR, XOR, AND, NOT / 1sC). S3 and S2 both high is a shift operation under S3, while S3 and S2 low signals an arithmetic operation. The precise operation being conducted within the modes is in turn defined by S1 and S0, with C being used in arithmetic mode to add 1.

\end{document}
